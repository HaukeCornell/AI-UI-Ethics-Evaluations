% Results Sect\begin{table}
\caption{Prevalence of Reasoning Patterns by Experimental Condition}
\label{tab:reasoning-patterns}
\begin{tabular}{lccccc}
\toprule
\textbf{Pattern} & \textbf{UEQ (\%)} & \textbf{UEEQ (\%)} & \textbf{RAW (\%)} & \textbf{$\chi^2$ p-value} & \textbf{Cramér's V} \\
\midrule
Responsibility Avoidance & 1.1 & 0.9 & 5.5 & <0.001 & 0.137 \\
Manipulation Awareness & 0.0 & 17.1 & 4.2 & <0.001 & 0.286 \\
Ethics-Focused Reasoning & 3.5 & 11.6 & 7.0 & <0.001 & 0.130 \\
Industry Conformity & 0.0 & 0.0 & 2.7 & <0.001 & 0.139 \\
Aesthetic Focus & 5.7 & 11.6 & 4.7 & <0.001 & 0.116 \\
Emotional Reaction & 0.7 & 3.5 & 1.7 & 0.008 & 0.086 \\
\bottomrule
\end{tabular}
\end{table}5 Paper
% Evaluation Framework Effects on Ethical Design Judgment

\section{Results}

Our analysis of 1,313 professional explanations across three evaluation conditions reveals systematic differences in reasoning patterns, providing quantitative evidence that evaluation frameworks significantly influence ethical design judgment.

\subsection{Overall Decision Patterns}
Participants made release decisions for 15 interface designs, resulting in an overall acceptance rate of 54.4\%. Decision patterns varied significantly across conditions: UEQ (55.9\% acceptance), UEEQ (52.6\% acceptance), and RAW (54.9\% acceptance). While overall acceptance rates showed modest differences, the underlying reasoning patterns revealed substantial variation.

\subsection{Reasoning Pattern Analysis}

Statistical analysis identified five distinct reasoning patterns with significant differences across experimental conditions (Table~\ref{tab:reasoning-patterns}).

\begin{table}
\caption{Prevalence of Reasoning Patterns by Experimental Condition}
\label{tab:reasoning-patterns}
\begin{tabular}{lccccc}
\toprule
\textbf{Pattern} & \textbf{UEQ (\%)} & \textbf{UEEQ (\%)} & \textbf{RAW (\%)} & \textbf{$\chi^2$ p-value} & \textbf{Cramér's V} \\
\midrule
Responsibility Avoidance & 1.1 & 0.9 & 5.5 & <0.001 & 0.137 \\
Manipulation Awareness & 0.0 & 16.7 & 3.5 & <0.001 & 0.289 \\
Industry Conformity & 0.0 & 0.0 & 2.7 & <0.001 & 0.139 \\
Aesthetic Focus & 5.7 & 11.6 & 4.7 & <0.001 & 0.116 \\
Emotional Reaction & 0.7 & 3.5 & 1.7 & 0.008 & 0.086 \\
\bottomrule
\end{tabular}
\end{table}

\subsubsection{Responsibility Avoidance}
The most striking difference appeared in responsibility avoidance patterns ($\chi^2 = 24.508$, $p < 0.001$, Cramér's V = 0.137). Participants in the RAW condition showed significantly higher rates of responsibility avoidance (5.5\%) compared to UEQ (1.1\%) and UEEQ (0.9\%) conditions. 

Representative examples from the RAW condition include:
\begin{quote}
\textit{``Given that the business team, marketing department, and my supervisor have all approved the design, and the development team has already started integrating it, withholding my final approval would be a significant professional and business risk.''}
\end{quote}

This pattern suggests that absence of quantitative evaluation data leads professionals to defer to institutional authority rather than making independent ethical judgments.

\subsubsection{Manipulation Awareness}
Manipulation awareness showed the largest effect size in our study ($\chi^2 = 109.424$, $p < 0.001$, Cramér's V = 0.289). Participants in the UEEQ condition demonstrated dramatically higher rates of manipulation recognition (16.7\%) compared to RAW (3.5\%) and UEQ (0.0\%) conditions.

Illustrative examples from UEEQ participants include:
\begin{quote}
\textit{``I HATE the language that is being used. It feels a little disrespectful and hateful. Needs to be a simple 'do you wish to remove profile' or something along those lines.''}
\end{quote}

\begin{quote}
\textit{``I would not release this interface because it is fundamentally deceptive: it tells users their geolocation is off while still using their location data. The deception score of -1.76 and attractiveness scores below -1.8 show users find it highly annoying and dishonest.''}
\end{quote}

This finding provides strong evidence that explicit ethical evaluation metrics enhance professionals' ability to identify and articulate manipulative design elements.

\subsubsection{Ethics-Focused Reasoning}
Explicit ethical reasoning showed significant differences across conditions ($\chi^2 = 22.078$, $p < 0.001$, Cramér's V = 0.130). The UEEQ condition demonstrated the highest prevalence of ethics-focused reasoning (11.6\%), compared to RAW (7.0\%) and UEQ (3.5\%) conditions.

Representative examples from UEEQ participants include:
\begin{quote}
\textit{``I chose not to release Interface 12 because the UX data reveals significant emotional and ethical risks.''}
\end{quote}

\begin{quote}
\textit{``I chose not to release Interface 13 because it violates core principles of ethical UX design.''}
\end{quote}

This pattern demonstrates that exposure to ethical evaluation frameworks not only increases recognition of problematic elements but also promotes explicit ethical reasoning and vocabulary usage.

\subsubsection{Industry Conformity}
Industry conformity justifications appeared exclusively in the RAW condition (2.7\%, $\chi^2 = 15.307$, $p < 0.001$, Cramér's V = 0.139). Participants without quantitative evaluation data frequently justified decisions by referencing industry standards:

\begin{quote}
\textit{``Common interface! Great UI, follows similarly to other social media giants.''}
\end{quote}

This pattern suggests that absence of specific evaluation criteria leads to reliance on industry precedent as a decision-making heuristic.

\subsubsection{Aesthetic Focus}
Aesthetic considerations were most prominent in the UEEQ condition (11.6\%), followed by UEQ (5.7\%) and RAW (4.7\%) conditions ($\chi^2 = 14.011$, $p < 0.001$, Cramér's V = 0.116). Interestingly, participants with access to ethical metrics showed increased attention to visual design elements, potentially indicating heightened overall design sensitivity.

\subsubsection{Emotional Reactions}
Emotional language appeared most frequently in UEEQ explanations (3.5\%) compared to RAW (1.7\%) and UEQ (0.7\%) conditions ($\chi^2 = 9.554$, $p = 0.008$, Cramér's V = 0.086). The presence of explicit ethical metrics appears to evoke stronger emotional responses to problematic designs.

\subsection{Topic Modeling Results}

Latent Dirichlet Allocation analysis identified eight primary topics in the explanation corpus, with significant distribution differences across conditions:

\begin{itemize}
\item \textbf{Business \& Commercial Focus} (24.8\% of explanations): Most prevalent in RAW condition
\item \textbf{Usability \& User Experience} (15.2\%): Evenly distributed across conditions  
\item \textbf{Technical \& Functional Issues} (12.7\%): Highest in UEEQ condition
\item \textbf{Ethical \& User Autonomy Concerns} (identified as distinct topic): Concentrated in UEEQ condition
\end{itemize}

\subsection{Effect of Evaluation Frameworks on Decision Quality}

Analysis of reasoning sophistication reveals that UEEQ participants demonstrated:
\begin{itemize}
\item Higher recognition of manipulative design elements (16.7\% vs. 0-3.5\% in other conditions)
\item More frequent use of ethical terminology and concepts
\item Greater emotional engagement with problematic designs
\item More detailed explanations referencing specific ethical concerns
\end{itemize}

Conversely, RAW participants showed:
\begin{itemize}
\item Increased delegation of responsibility to institutional authority (5.5\% vs. ~1\% in other conditions)
\item Greater reliance on industry conformity as justification (2.7\% vs. 0\% in other conditions)
\item Less sophisticated ethical reasoning overall
\end{itemize}

\subsection{Statistical Significance and Effect Sizes}

All five reasoning patterns showed statistically significant differences across conditions ($p < 0.01$). Effect sizes ranged from small (Emotional Reaction, V = 0.086) to large (Manipulation Awareness, V = 0.289), indicating both statistical significance and practical importance.

The manipulation awareness pattern showed the largest effect size (V = 0.289), suggesting that explicit ethical metrics have a substantial impact on professionals' ability to identify problematic design elements. Responsibility avoidance (V = 0.137) and industry conformity (V = 0.139) showed medium effect sizes, indicating meaningful differences in decision-making approaches between conditions.
