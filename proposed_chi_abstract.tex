% Proposed CHI 2025 Abstract - Under 150 words

User experience evaluation frameworks shape how designers perceive interface designs, yet their influence on ethical design judgment remains unexplored. Through a three-condition experiment with 141 UX professionals evaluating 15 social media dark patterns, we demonstrate that evaluation frameworks systematically affect designers' willingness to implement manipulative interfaces.

Using actual user evaluation data collected from 120 social media users, we presented dark pattern interfaces under three conditions: standard User Experience Questionnaire (UEQ) metrics, UEQ enhanced with autonomy-focused measures, or no evaluation data. Results revealed a clear progression in design acceptance: standard UEQ metrics led to highest dark pattern acceptance (M=3.80), no evaluation was moderate (M=4.66), while autonomy-enhanced evaluation produced strongest rejection (M=3.14). Effect sizes were substantial (d=0.56-1.20), with rejection rates nearly doubling from 30% to 56%.

These findings suggest standard UX metrics may inadvertently legitimize manipulative design by emphasizing usability while obscuring ethical concerns. This work reveals that evaluation frameworks actively influence ethical design decisions, calling for assessment methods aligned with societal values.

% Keywords: Dark Patterns, User Experience Evaluation, Ethical Design, Design Decision-Making, UX Metrics
